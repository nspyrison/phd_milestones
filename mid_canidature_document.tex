\documentclass[11,]{article}
\usepackage{lmodern}
\usepackage{amssymb,amsmath}
\usepackage{ifxetex,ifluatex}
\usepackage{fixltx2e} % provides \textsubscript
\ifnum 0\ifxetex 1\fi\ifluatex 1\fi=0 % if pdftex
  \usepackage[T1]{fontenc}
  \usepackage[utf8]{inputenc}
\else % if luatex or xelatex
  \ifxetex
    \usepackage{mathspec}
  \else
    \usepackage{fontspec}
  \fi
  \defaultfontfeatures{Ligatures=TeX,Scale=MatchLowercase}
\fi
% use upquote if available, for straight quotes in verbatim environments
\IfFileExists{upquote.sty}{\usepackage{upquote}}{}
% use microtype if available
\IfFileExists{microtype.sty}{%
\usepackage{microtype}
\UseMicrotypeSet[protrusion]{basicmath} % disable protrusion for tt fonts
}{}
\usepackage[margin=1in]{geometry}
\usepackage{hyperref}
\PassOptionsToPackage{usenames,dvipsnames}{color} % color is loaded by hyperref
\hypersetup{unicode=true,
            colorlinks=true,
            linkcolor=Maroon,
            citecolor=Blue,
            urlcolor=blue,
            breaklinks=true}
\urlstyle{same}  % don't use monospace font for urls
\usepackage{longtable,booktabs}
\usepackage{graphicx,grffile}
\makeatletter
\def\maxwidth{\ifdim\Gin@nat@width>\linewidth\linewidth\else\Gin@nat@width\fi}
\def\maxheight{\ifdim\Gin@nat@height>\textheight\textheight\else\Gin@nat@height\fi}
\makeatother
% Scale images if necessary, so that they will not overflow the page
% margins by default, and it is still possible to overwrite the defaults
% using explicit options in \includegraphics[width, height, ...]{}
\setkeys{Gin}{width=\maxwidth,height=\maxheight,keepaspectratio}
\usepackage[normalem]{ulem}
% avoid problems with \sout in headers with hyperref:
\pdfstringdefDisableCommands{\renewcommand{\sout}{}}
\IfFileExists{parskip.sty}{%
\usepackage{parskip}
}{% else
\setlength{\parindent}{0pt}
\setlength{\parskip}{6pt plus 2pt minus 1pt}
}
\setlength{\emergencystretch}{3em}  % prevent overfull lines
\providecommand{\tightlist}{%
  \setlength{\itemsep}{0pt}\setlength{\parskip}{0pt}}
\setcounter{secnumdepth}{5}
% Redefines (sub)paragraphs to behave more like sections
\ifx\paragraph\undefined\else
\let\oldparagraph\paragraph
\renewcommand{\paragraph}[1]{\oldparagraph{#1}\mbox{}}
\fi
\ifx\subparagraph\undefined\else
\let\oldsubparagraph\subparagraph
\renewcommand{\subparagraph}[1]{\oldsubparagraph{#1}\mbox{}}
\fi

%%% Use protect on footnotes to avoid problems with footnotes in titles
\let\rmarkdownfootnote\footnote%
\def\footnote{\protect\rmarkdownfootnote}

%%% Change title format to be more compact
\usepackage{titling}

% Create subtitle command for use in maketitle
\providecommand{\subtitle}[1]{
  \posttitle{
    \begin{center}\large#1\end{center}
    }
}

\setlength{\droptitle}{-2em}

  \title{}
    \pretitle{\vspace{\droptitle}}
  \posttitle{}
    \author{}
    \preauthor{}\postauthor{}
    \date{}
    \predate{}\postdate{}
  
%%% .rmd + .sty setup borrowed from: https://github.com/oganm/ThesisProposal

% Must load tex packages here (import.sty run in preamble, title.sty run after)
\usepackage{setspace} % for title page spacing
\usepackage{hyperref} % for all sorts of linking

\begin{document}

%%% .rmd + .sty setup borrowed from: https://github.com/oganm/ThesisProposal

\onehalfspacing
\pagenumbering{gobble}

%\begin{titlepage}
\begin{center}
\LARGE{\textbf{Identification cell type marker genes of the brain and their use in estimation of cell type proportions}}\\
\vspace*{2\baselineskip}
\Large{\textbf{Mid canidature review}}\\
\normalsize{Monash University, Faculty of Information Technology}\\
\vspace*{2\baselineskip}
\Large{Nicholas Spyrison}\\ %, B.Sc
\vspace*{3\baselineskip}
\Large{\textbf{Thesis Supervisors}}\\
Prof. Kimbal Marriott\\
Prof. Dianne Cook\\
\vspace*{2\baselineskip}
\Large{\textbf{Committee Members}}\\
Dr. Maxime Cordiel\\
Dr. Shirui Pan\\
Dr. Sara Mostafavi\\
\vspace*{1\baselineskip}
\Large{\textbf{Chair}}\\
Assoc. Prof. Bernhard Jenny\\
\vspace*{1\baselineskip}
\Large{\textbf{Presention Date}}\\
DD Feburuary, 2020
\end{center}
% \end{titlepage}

\doublespacing

\hypersetup{linkcolor = blue}
\newpage
\pagenumbering{roman}
\tableofcontents
\addcontentsline{toc}{section}{\contentsname}

\newpage

%% list of figures have to be added manually to table of contents
% \listoffigures 
% 
% \newpage
% \listoftables

\doublespacing

\newpage
\pagenumbering{arabic}
\hypersetup{linkcolor = blue}

{
\hypersetup{linkcolor=black}
\setcounter{tocdepth}{2}
\tableofcontents
}
\hypertarget{sec:intro}{%
\section{Rmarkdown note}\label{sec:intro}}

\hypertarget{references-and-cross-references}{%
\subsection{References and cross references}\label{references-and-cross-references}}

A bib reference (Wickham, Cook, and Hofmann 2015).

A \protect\hyperlink{sec:intro}{Section intro} reference, alternatively, section \ref{sec:intro} (with no @; \textbackslash ref\{sec:intro\}).

A figure \ref{fig:crest} reference (with @; \ref{fig:crest}.

\textbf{For bib files:} make sure to include \texttt{bibliography:\ \textless{}MyFilename\textgreater{}.bib} in the YAML header and have asscociated .bib file in the base directory.

\textbf{For Tex Title pages:} make sure to include them in the YAML header (see this one) and have the asscociated .sty files in the base directory.

\textbf{For figure references:} make sure to use \texttt{bookdown::pdf\_document2} (with 2) in the yaml, turns an R chunk name into a label, alternatively hardcode labels in the caption with \texttt{"\textbackslash{}\textbackslash{}label\{fig:fig1\}\ Some\ caption\ here"} while kniting to a \texttt{pdf\_document}.

\hypertarget{sec2:subsection}{%
\subsection{Figure height and width}\label{sec2:subsection}}

\texttt{fig.height=2,\ fig.width=3} {[}inches{]} will effect R rendered plots.

while \texttt{out.height\ =\ "20\%",\ out.width\ =\ "20\%"} will effect both \texttt{include\_graphics()} and R rendered plots.

\begin{figure}

{\centering \includegraphics[width=0.2\linewidth,height=0.2\textheight]{figures/can_con/crest} 

}

\caption{A caption for crest figure}\label{fig:crest}
\end{figure}

\hypertarget{intermediate-md-formating}{%
\subsection{Intermediate MD formating}\label{intermediate-md-formating}}

sub/superscript\textsuperscript{2}\textsubscript{2}

\sout{strikethrough}

escaped characters: * \_ \textbackslash{}

endash: --, emdash: ---

in-line equation: \(A = \pi*r^{2}\)

equation block: \[E = mc^{2}\]

\begin{quote}
``If it weren't for my lawyer, I'd still be in prison.
It went a lot faster with two people digging.''

\hfill --- Joe Martin
\end{quote}

A footnote\footnote{The bottom text.}

\hypertarget{other-sources}{%
\subsubsection{Other sources}\label{other-sources}}

\href{http://zevross.com/blog/2017/06/19/tips-and-tricks-for-working-with-images-and-figures-in-r-markdown-documents/}{Working with images and figure in RMD}

\href{https://rstudio.com/wp-content/uploads/2016/03/rmarkdown-cheatsheet-2.0.pdf}{R Markdown Cheat Sheet}

\hypertarget{references}{%
\section*{References}\label{references}}
\addcontentsline{toc}{section}{References}

\hypertarget{refs}{}
\leavevmode\hypertarget{ref-wickham_visualizing_2015}{}%
Wickham, Hadley, Dianne Cook, and Heike Hofmann. 2015. ``Visualizing Statistical Models: Removing the Blindfold: Visualizing Statistical Models.'' \emph{Statistical Analysis and Data Mining: The ASA Data Science Journal} 8 (4): 203--25. \url{https://doi.org/10.1002/sam.11271}.


\end{document}
